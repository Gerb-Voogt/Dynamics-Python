\documentclass[11pt, a4paper]{article}

\usepackage{graphicx}
\usepackage[english]{babel}
\usepackage[utf8x]{inputenc}
\usepackage{amsmath}
\usepackage[a4paper,top=3cm,bottom=2cm,left=2cm,right=2cm,marginparwidth=1.75cm]{geometry}
\usepackage{amssymb}

\graphicspath{ {./images} }

\makeatletter
\renewcommand*\env@matrix[1][*\c@MaxMatrixCols c]{%
  \hskip -\arraycolsep
  \let\@ifnextchar\new@ifnextchar
  \array{#1}}
\makeatother

\begin{document}
\setcounter{section}{1}
\section{Lecture 2: Spans and vector equations (14/02/2020)}

\subsection{vectors in $\mathbb{R}^n$ space}
Any vector in $\mathbb{R}^n$ space is denoted with the following form:

\begin{align*}
    \vec{v} =
    \begin{pmatrix}
        v_1\\
        v_2\\
        \vdots \\
        v_n\\
    \end{pmatrix}
\end{align*}
The important algebraic properties of any vector in $\mathbb{R}^n$ space will be listed below.
\begin{gather}
    \vec{u} + \vec{v} = \vec{v} + \vec{u}\\
    \vec{u} + -\vec{u} = -\vec{u} + \vec{u} = 0\\
    c(\vec{u} + \vec{v}) = c\vec{u} + c\vec{v}\\
    (c + d)\vec{u} = c\vec{u} + d\vec{u}\\
    c(d\vec{u}) = cd\vec{u}
\end{gather}
It is also worth knowing that a vector can be divided by it's own magnitude in which case all of
it's components become smaller then 1. This proces is called normalization and creates a vector of size 1
which only represents the direction of said vector.

\begin{align*}
    \hat{u} = \frac{\vec{u}}{|\vec{u}|}
\end{align*}

\subsection{vector equations}
Vector equations are a different way of representing a given linear system. they take the following form:
\begin{gather*}
    \vec{y} = c_1\vec{v}_1 + c_2\vec{v}_2 + \cdots + c_n\vec{v}_n\\
    \vec{y} = \sum_{i=1}^{n} c_i\vec{v}_i
\end{gather*}
$\vec{y}$ is refered to as the linear combination of $\vec{v}_1 \cdots \vec{v}_n$. The $c_i$ terms are referred
to as the weight.\\
The vector equation for $\vec{y}$ can also be represented as a $1\times n$ matrix as follows:
\begin{align*}
    \begin{bmatrix}[cccc|c]
        \vec{v}_1 & \vec{v}_2 & \cdots & \vec{v_n} & \vec{y}\\
    \end{bmatrix}
\end{align*}
\\
As mentioned before linear systems can also be expressed as vector equations.
This looks like the following:
\begin{align*}
    \begin{cases} 
        x_1 + 5x_2 + 3x_3 &= 1 \\
        2x_1 + x_2 + 15x_3 &= 8 \\
        \end{cases}
    \quad &\Leftrightarrow \quad
    \begin{pmatrix}[cc]
        2x_1 + 3x_2 \\
        -x_1 + 2x_2 \\
    \end{pmatrix}
        =
    \begin{pmatrix}
        7\\
        0\\
    \end{pmatrix}
\end{align*}
Which can be rewritten as:
\begin{align*}
    x_1 \cdot
    \begin{pmatrix}
        2\\
        -1\\
    \end{pmatrix}
    + x_2 \cdot
    \begin{pmatrix}
        3\\
        2\\
    \end{pmatrix}
    =
    \begin{pmatrix}
        7\\
        0\\
    \end{pmatrix}
\end{align*}
The advantage of vector equations is that they are easily graphically interpretable in 2 or 3 dimensional space.
They can also give information on the possible solution of a given system of equation. When lines
are parrallel they have no solution. When lines cross in a singular point the system has a unique solution.

\subsection{Vector spans}
given a set of vectors $\vec{v}_1, \vec{v}_2,\cdots, \vec{v}_n$ in $\mathbb{R}^n$.
The set of all linear combinations is denoted by:
\begin{align*}
    \text{Span} 
    \begin{Bmatrix}
        \vec{v}_1, &\vec{v_2}, & \cdots, & \vec{v}_n \\
    \end{Bmatrix}
\end{align*}
The span is the subset of all vectors that can be written as $\sum_{i=1}^{n} c_i\vec{v}_i$. This means the span
is just the collection of any given point in an  $\mathbb{R}^n$ space. that can be reached with the given vectors.

\begin{align*}
    \text{Span} 
    \begin{Bmatrix}
        \vec{v}_1, & \vec{v}_2 \\
    \end{Bmatrix}
    \hspace{1em} \text{where} \hspace{1em} \vec{v}_1 =
    \begin{pmatrix}
        1\\
        0\\
    \end{pmatrix}
    \; \text{and} \; \vec{v}_2 = 
    \begin{pmatrix}
        2\\
        0\\
    \end{pmatrix}\\
    \text{All possible vectors are of the form: }
    \begin{pmatrix}
        a\\
        0\\
    \end{pmatrix}
    \: ,\, a \in \mathbb{R}
\end{align*}
Graphically this means that all possible values of the linear combination 
of these vectors are somewhere on a horizontal line through the origin.
When instead the following vectors where given:
\begin{align*}
    \vec{v}_1 =
    \begin{pmatrix}
        1\\
        0\\
    \end{pmatrix}
    \; \text{and} \; \vec{v}_2 =
    \begin{pmatrix}
        0\\
        1\\
    \end{pmatrix}
\end{align*}
The span would be all possible points on a 2 dimensional plane, or represented as a vector:
\begin{align*}
    \begin{pmatrix}
        a\\
        b\\
    \end{pmatrix}
    \, , \; a, b \in \mathbb{R}
\end{align*}
The span of a vector can either be an entire plane, a line or a single point in any $\mathbb{R}^n$ space\footnote{
    Or something more abstract when considering a vector space with more then 3 dimensions since this cannot graphically interpreted.}.
To reach any given point in $\mathbb{R}^n$ space atleast $n$ vectors are required.


\end{document}