\documentclass[11pt, a4paper]{article}

\usepackage{graphicx}
\usepackage[english]{babel}
\usepackage[utf8x]{inputenc}
\usepackage{amsmath}
\usepackage[a4paper,top=3cm,bottom=2cm,left=2cm,right=2cm,marginparwidth=1.75cm]{geometry}
\usepackage{amssymb}

\graphicspath{ {./images} }

\makeatletter
\renewcommand*\env@matrix[1][*\c@MaxMatrixCols c]{%
  \hskip -\arraycolsep
  \let\@ifnextchar\new@ifnextchar
  \array{#1}}
\makeatother

\newcommand*{\qed}{\hfill\ensuremath{\quad\square}}%

\begin{document}

\setcounter{section}{5}
\section{Lecture 6: Matrix Operations (28/02/2020)}
\subsection{Types of matrices}
There are several different types of matrices, described by there entries and rank. These are as follows:
\begin{itemize}
  \item Zero matrix, all entries are 0
  \item Square matrix, $m\times n$ matrix where $m = n$
  \item Diagonal matrix, matrix containing only zeros and main entries
  \item Identity matrix, diagonal matrix where all main entries are 1
  \item Lower/Upper triangular matrix, a square matrix in echelon form.
\end{itemize}

Examples of these are\footnote{These examples use a $3 \times 3$ matrix since it's easy to fit on the page for an example, but the matrices
can be any $m \times n$ dimension.}: 
\begin{align*}
  \text{Zero Matrix: }
  \begin{bmatrix}
    0 & 0 & 0\\
    0 & 0 & 0\\
    0 & 0 & 0\\
  \end{bmatrix}&
  \qquad \text{Diagonal matrix: }
  \begin{bmatrix}
    \blacksquare & 0 & 0\\
    0 & \blacksquare & 0\\
    0 & 0 & \blacksquare\\
  \end{bmatrix}
  \\
  \text{Identity: }
  \begin{bmatrix}
    1 & 0 & 0\\
    0 & 1 & 0\\
    0 & 0 & 1\\
  \end{bmatrix}&
  \qquad \text{Upper triangular matrix: }
  \begin{bmatrix}
    \blacksquare & * & *\\
    0 & \blacksquare & *\\
    0 & 0 & \blacksquare\\
  \end{bmatrix}
  \\
\end{align*}

\subsection{Matrix multiplication}
\underline{Theorem:}
Given two linear transformations $T: \mathbb{R}^p \rightarrow \mathbb{R}^n$ and 
$S: \mathbb{R}^n \rightarrow \mathbb{R}^m$, then the composite $S \circ T: \mathbb{R}^p \rightarrow \mathbb{R}^m$,
defined by $(S \circ T)(\vec{x}) = S(T(\vec{x}))$ is also a linear transformation.\\
\\
Proof that $S \circ T$ is indeed linear:\\
For a transformation to be linear it needs to statisfy the following 2 properties:
\begin{enumerate}
  \item $T(\vec{u} + \vec{v}) = T(\vec{u}) + T(\vec{v})$
  \item $T(c\vec{u}) = cT(\vec{u})$
\end{enumerate}
let $T: \mathbb{R}^p \rightarrow \mathbb{R}^n$ and $S: \mathbb{R}^n \rightarrow \mathbb{R}^m$ be linear transformations, 
let $\vec{u}$ and $\vec{v}$ be vectors and let $c$ be a scalar.
\begin{align*}
  (S \circ T)(\vec{u} + \vec{v}) &= S(T(\vec{u}) + T(\vec{v}))\\
                                 &= S(T(\vec{u})) + S(T(\vec{v}))\\
                                 &= (S \circ T)(\vec{u}) + (S \circ T)(\vec{v}) \qed
\end{align*}
\begin{align*}
  (S \circ T)(c\vec{u}) = S(cT(\vec{u})) &= cS(T(\vec{u}))\\
                                         &= c(S \circ T)(\vec{u}) \qed
\end{align*}


The matrix multiplication of the matrices $A: m\times n$ and $B: n\times p$ is equal to the standard
matrix of the composite mapping $S \circ T$, where $S(\vec{x}) = A\vec{x}$ and $T(\vec{x}) = B\vec{x}$.
Note that $AB$ will be an $m\times p$ matrix.
In general matrix multiplication takes the following form:
\begin{align*}
  &S: \mathbb{R}^n \rightarrow \mathbb{R}^m \quad \text{Where} \quad S(\vec{x}) = A\vec{x} \quad \text{and} \quad A: m \times n\\
  &T: \mathbb{R}^p \rightarrow \mathbb{R}^n \quad \text{Where} \quad T(\vec{y}) = B\vec{y} \quad \text{and} \quad B: n \times p\\
  &S \circ T: \mathbb{R}^p \rightarrow \mathbb{R}^m \quad \text{and} \quad A\cdot B : [m \times n][n \times p]\\
\end{align*}

What this means is that $S$ is a linear transformation from $\mathbb{R}^n$ to $\mathbb{R}^m$ and $T$ a linear transformation from
$\mathbb{R}^p$ to $\mathbb{R}^m$. The matrix product associated with that linear transformation will be a matrix product of $A$ with dimension $n \times m$ and a matrix
$B$ with dimensions $n \times p$, which will output a single matrix of dimensions $m \times p$.
\\
\\
Computing a specific entry of a matrix $AB$ can be done with the row column rule. The most general form of matrix multiplication looks like the following:
entry $(i, j)$ of $AB$ equals $\text{row}_i(A)b_j$.

\begin{align*}
  (AB)_{ij} &= a_{1i}b_{1j} + a_{i2}b_{2j} + \cdots + a_{in}b_{nj}\\
            &= \sum_{k=1}^{n} a_{ik}b_{kj}
\end{align*}

\begin{align*}
  \begin{bmatrix}
    a_{11} & \cdots & a_{1j} & \cdots & a_{1n}\\
    \vdots & \,     & \vdots & \,     & \vdots\\
    a_{i1} & \cdots & a_{ij} & \cdots & a_{in}\\
    \vdots & \,     & \vdots & \,     & \vdots\\
    a_{m1} & \cdots & a_{mj} & \cdots & a_{mn}\\
  \end{bmatrix}
  \begin{bmatrix}
    b_{11} & \cdots & b_{1j} & \cdots & b_{1p}\\
    \vdots & \,     & \vdots & \,     & \vdots\\
    b_{i1} & \cdots & b_{ij} & \cdots & b_{ip}\\
    \vdots & \,     & \vdots & \,     & \vdots\\
    b_{n1} & \cdots & b_{nj} & \cdots & b_{np}\\
  \end{bmatrix}
\end{align*}

There are some important algebraic properties to consider for matrix multiplication.
Most of these will seem familiar since they look a lot like regular algebraic expressions, however there is
1 key difference, mainly that matrices are non-communative. The rules for manipulating matrices algebraicly are as follow:
\begin{gather}
  AB \neq BA\\
  A(BC) = (AB)C\\
  A(B+C) = AB + AC\\
  (B+C)A = BA + CA\\
  I_mA = A = AI_n\\
  r(AB) = (rA)B = A(rB)\\
  A^k = AA \cdots A \text{where A is repeated $k$ times}\\
  A^0 = I
\end{gather}

Some common mistakes that are made:
\begin{gather}
  AB = AC \quad \text{does not imply that $B = C$.}\\
  AB = 0 \quad \text{does not imply that either $A$ or $B$ is always a zero-matrix, thought they could be.}\\
  (A + B)^2 = A^2 + AB + BA + B^2 \quad \text{note that $AB \neq BA$}
\end{gather}

\subsection{Numerical example of matrix multiplication}
let $A$ be an $2 \times 3$ matrix and let $B$ be a $3 \times 2$ matrix of the following form:
\begin{align*}
  \begin{bmatrix}
    1  & 2 & 3\\
    -1 & 0 & 2\\
  \end{bmatrix}
  \cdot
  \begin{bmatrix}
    0 & 1\\
    1 & -1\\
    0 & 2\\
  \end{bmatrix}
  =
  \begin{bmatrix}
    0 + 2 + 0 & 1 - 2 + 6\\
    0 + 0 + 0 & -1 +0 + 4\\
  \end{bmatrix}
  =
  \begin{bmatrix}
    2 & 5\\
    0 & 3\\
  \end{bmatrix}
\end{align*}
Recall that this is the same as repeated matrix-vector multiplication, and again
the same as the dot product between a column of the second matrix and a row of the first matrix.

\subsection{Transpose of a matrix}
Might write this here, or might write it in Lecture 7 Notes haven't decided yet.

\end{document}