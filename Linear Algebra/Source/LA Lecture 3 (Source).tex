\documentclass[11pt, a4paper]{article}

\usepackage{graphicx}
\usepackage[english]{babel}
\usepackage[utf8x]{inputenc}
\usepackage{amsmath}
\usepackage[a4paper,top=3cm,bottom=2cm,left=2cm,right=2cm,marginparwidth=1.75cm]{geometry}
\usepackage{amssymb}

\graphicspath{ {./images} }

\makeatletter
\renewcommand*\env@matrix[1][*\c@MaxMatrixCols c]{%
  \hskip -\arraycolsep
  \let\@ifnextchar\new@ifnextchar
  \array{#1}}
\makeatother

\begin{document}

\setcounter{section}{2}
\section{Lecture 3: Matrix-vector products and solution sets (18/02/2020)}

\subsection{general form of vectors and matrices}
Matrices can be thought of an $m \times n$ grid of numbers. A column vector is thus by extension
of the form $1 \times n$. let the $\vec{x}$ be a random column vector and $A$ a $m \times 1$ matrix.  
Then: 
\begin{align*}
  A = 
  \begin{bmatrix}[cccc]
    a_1 & a_2 & \cdots & a_m\\
  \end{bmatrix}
  \quad \text{and} \quad 
  \vec{x} = 
  \begin{bmatrix}
    x_1\\
    x_2\\
    \vdots \\
    x_n\\
  \end{bmatrix}
\end{align*}
\\
The matrix vector product is defined as the following:

\begin{align*}
  \vec{y} = A\vec{x} =
  \begin{pmatrix}
    a_{11}x_1 + \cdots + a_{1n}x_n\\
    a_{21}x_1 + \cdots + a_{2n}x_n\\
    \vdots\\
    a_{m1}x_1 + \cdots + a_{mn}x_n\\
  \end{pmatrix}
\end{align*}
\\
It is thus possible to compute the matrix-vector product using this definition, it's worth noting that
$A\vec{x}$ of the form $m \times n$ will always be of the form $\vec{x} \in \mathbb{R}^n$ and 
$\vec{y} = \mathbb{R}^m$. Which is to say, The matrix vector product of an $m \times n$ 
matrix and vector of $n$ entries will output
a vector with $m$ entries.\\
\\
A quick numerical example will follow:
\begin{align*}
  \begin{bmatrix}[cc]
    2 & 0\\
    3 & 2\\
    1 & 2\\
  \end{bmatrix}
  \cdot
  \begin{pmatrix}
    1\\
    3\\
  \end{pmatrix}
  =
  1 \cdot \begin{pmatrix} 2\\ 3\\ 1\\ \end{pmatrix} +
  3 \cdot \begin{pmatrix} 0\\ 2 \\ 2\\ \end{pmatrix} =
    \begin{pmatrix} 2+0\\ 3+6\\ 1+6\\ \end{pmatrix}
  =
  \begin{pmatrix} 1\\ 9\\ 7\\ \end{pmatrix}
\end{align*}

\subsection{Algebraic rules for matrix vector multiplication}
There are some important algebraic rules to note for matrix-vector multiplication. All properties
relating to manipulation of matrix-vector multiplication can be derrived using these 2 definitions.
let $A$ be an $m \times n$ matrix, $\vec{v}$ and $\vec{u}$ $1 \times n$ vectos and $c$ and $d$ a scalar.
\begin{gather}
  A(\vec{u} + \vec{v}) = A\vec{u} + A\vec{v}\\
  A(c\vec{u}) = c(A\vec{u})
\end{gather}
Combining equation (1) and (2) gives the following, which is also sometimes given as a third rule:
\begin{gather}
  A(c\vec{u} + d\vec{v}) = c(A\vec{u}) + d(A\vec{v})
\end{gather}

It's worth noting that matrix-multiplication can alternetivly be interpreted as the dot product
between a row in the matrix and the column vector $\vec{v}$. Thinking about it this way can save time
when computing matrix-vector products.

\subsection{Notations for matrix-vector products}
All the following forms are equavalant: 
\begin{align*}
  \begin{cases}
    x + 2y + 3z =3\\
    2x - y -2z = 4\\
    -x + y + 3z = 5\\
  \end{cases}
  \Leftrightarrow
  x \cdot \begin{pmatrix} 1\\ 2\\ -1\\ \end{pmatrix} +
    y \cdot \begin{pmatrix} 2\\ -1\\ 1\\ \end{pmatrix} +
    z \cdot \begin{pmatrix} 3\\ -2\\ 3\\ \end{pmatrix} =
    \begin{pmatrix} 3\\ 4\\ -5\\ \end{pmatrix}
  \Leftrightarrow
  \begin{bmatrix}
    1 & 2 & 3\\
    2 & -1 & -2\\
    -1 & 1 & 3\\
  \end{bmatrix}
  \cdot
  \begin{pmatrix} x\\ y\\ z\\ \end{pmatrix}
  =
  \begin{pmatrix} 3\\ 4\\ -5\\ \end{pmatrix}
\end{align*}

\subsection{Homogenous and Non-homogenous systems}
Is system is defined to be homogenous if $A\vec{x} = \vec{y}$ where $\vec{y} = 0$.
Conversly a system is defined to be non-homogenous if $A\vec{x} = \vec{y}$ where $\vec{y} \neq 0$.
Homogenous systems have some interesting properties. These are as follows:
\begin{itemize}
  \item Homogenous systems are always consistent, and thus always have a solution
  \item Homogenous systems have a non-trivial\footnote{a trvial solution is a solution where all variables are 0. It's technically true but doesn't tell us anything about the system.} 
        solution if and only if it has 1 free variable.
\end{itemize}

Some examples of homogenous and non-homogenous systems will follow:
\begin{align*}
  \begin{bmatrix}[ccc|c]
    3 & 4 & -5 & 0\\
    -3 & -2 & 1 & 0\\
    6 & 1 & 4 & 0\\
  \end{bmatrix}
  \sim
  \begin{bmatrix}
    1 & 0 & 1 & 0\\
    0 & 1 & -2 & 0\\
    0 & 0 & 0 & 0\\
  \end{bmatrix}
  \Rightarrow
  \begin{pmatrix} x_1 \\ x_2 \\ x_3 \end{pmatrix}
  = \begin{pmatrix} -x_3\\ 2x_3\\ x_3\\ \end{pmatrix}
  = x_3 \cdot \begin{pmatrix} -1\\ 2\\ 1\\ \end{pmatrix}
\end{align*}

\begin{align*}
  \begin{bmatrix}[ccc|c]
    3 & 4 & -5 & 1\\
    -3 & -2 & 1 & 1\\
    6 & 1 & 4 & -5\\
  \end{bmatrix}
  \sim
  \begin{bmatrix}
    1 & 0 & 1 & -1\\
    0 & 1 & -2 & 1\\
    0 & 0 & 0 & 0\\
  \end{bmatrix}
  \Rightarrow
  \begin{pmatrix} x_1 \\ x_2 \\ x_3 \end{pmatrix}
  = \begin{pmatrix} -1 - x_3\\ 1 + 2x_3\\ x_3\\ \end{pmatrix}
  = \begin{pmatrix} -1\\ 1\\ 0\\ \end{pmatrix} + x_3 \cdot \begin{pmatrix} -1\\ 2\\ 1\\ \end{pmatrix}
\end{align*}
\\
Note that the part depending on the free variable does not change between homogenous 
and non-homogenous systems. Graphically this ca be interpreted as shifting the line of 
infinite solutions away  from the origin by some vector. For this example that vector is 
$\langle -1, 1, 0 \rangle$. Because of this general solution to a non-homogenous system can
be written in the form  $\vec{x} = \vec{x}_p + \vec{x}_h$., where $\vec{x}_p$ is the particular
solution and $\vec{x}_h$ the homogenous solution\footnote{This looks very similar to the solution of a 
non-homogenous differential equation (Analyse 1)}. 
It can be proven that these are any and all solutions to the system. (maybe insert proof later idk yet)

\end{document}