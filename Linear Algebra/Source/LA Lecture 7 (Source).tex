\documentclass[11pt, a4paper]{article}

\usepackage{graphicx}
\usepackage[english]{babel}
\usepackage[utf8x]{inputenc}
\usepackage{amsmath}
\usepackage[a4paper,top=3cm,bottom=2cm,left=2cm,right=2cm,marginparwidth=1.75cm]{geometry}
\usepackage{amssymb}
\usepackage{subfig}

\graphicspath{ {./images} }

\makeatletter
\renewcommand*\env@matrix[1][*\c@MaxMatrixCols c]{%
  \hskip -\arraycolsep
  \let\@ifnextchar\new@ifnextchar
  \array{#1}}
\makeatother

\newcommand*{\qed}{\hfill\ensuremath{\quad\square}}%
\newcommand*{\rad}{\ensuremath{\,\text{rad}}}

\begin{document}
\setcounter{section}{6}
\section{Lecture 7: Transposed matrices and inverted matrices (03/03/2020)}
\subsection{Transposed matrices}
A transpose is a type of matrix operation where the rows and columns of a matrix are interchanged.
This takes the following form for a $2 \times 2$ matrix, but can be extenden to any $m \times n$ matrix.

\begin{align*}
  A = 
  \begin{bmatrix}
    a_{11} & a_{12}\\
    a_{21} & a_{22}\\
  \end{bmatrix}
  \quad \text{Then} \quad 
  A^T = 
  \begin{bmatrix}
    a_{11} & a_{21}\\
    a_{12} & a_{22}\\
  \end{bmatrix}
\end{align*}

Some algebraic rules for matrix transposition are:
\begin{gather}
  (A^T)^T = A\\
  (A + B)^T = A^T + B^T\\
  (rA)^T = rA^T\\
  (AB)^T = B^TA^T
\end{gather}

Quick proof for equation (1):
\begin{gather*}
  (A^T)^T = 
  \begin{bmatrix}
    a_{11} & a_{21}\\
    a_{12} & a_{22}\\
  \end{bmatrix} ^T
  =
  \begin{bmatrix}
    a_{11} & a_{12}\\
    a_{21} & a_{22}\\
  \end{bmatrix}
  = A \qed
\end{gather*}
It's worth noting that equation (4) may not intuitivly make sensen based on regular algebraic rules, however
when we consider that the columns in $B$ must be equal to the rows in $A$ for matrix multiplication to make sense\
it's easy to see that the order $B$ and then $A$ makes sense. 
if $A$,$B$ are $m \times n$ and $n \times m$ matrices respectivly, the transposed matrices $A^T$ and $B^T$ must be of the form $n \times m$ and $p \times n$.

\subsection{Inverse matrices}
\underline{Theorem:}
if $A$ is an invertible $m \times n$ matrix, then for each $\vec{b}$ in $\mathbb{R}^n$ the equation
$A\vec{x}=\vec{b}$ has the unique solution $\vec{x} = A^{-1}\vec{b}$.

\begin{figure}[h]
  \centering
  \subfloat[$A^{-1}$ doesn't exist]{{\includegraphics[width=50mm]{images/placeholder.png}}}%
  \qquad
  \subfloat[$A^{-1}$ does exist]{{\includegraphics[width=50mm]{images/placeholder.png}}}%
  \caption{Representation of the existence of an inverse matrix in 2-dimensional space}
\end{figure}
Note figure 1 (a). There is no single unique vector $\vec{v}$ that will project to the x-axis under the
given matrix transformation, rather there is an entire line of vectors which will give the same solution,
making the solution non-unique and by extension the matrix non-invertable. Figure 1 (b) represents a rotation
of space by $\frac{\pi}{2}\rad$. The transfomation of vector will always be unique, making the matrix invertible.\\
\\
Some rules to consider for inverting matrices:
\begin{gather}
  (A^{-1})^{-1} = A\\
  (AB)^{-1} = B^{-1}A^{-1}\\
  (A^T)^{-1} = (A^{-1})^T
\end{gather}

Proof for equation (6) and (7):
\begin{gather*}
  (AB)^{-1} \cdot AB = I_n\\
  B^{-1}A^{-1} \cdot AB = B^{-1}I_nB = B^{-1}B = I_n \qed\\
  \\
  (A^T)^{-1} A^T = I_n\\
  (A^{-1})A^T = (AA^{-1})^T = I_n^T = I_n \qed
\end{gather*}

\underline{Theorem:}
A matrix $A: n \times n$ is invertible if and only if $A$ is row equivelant to $I_n$.\\

This means the matrix $A$ should have exactly $n$ pivots since there can be no free variables in
an invertable matrix.

\subsection{Special case of a $2 \times 2$ matrix}
Computing the inverse matrix of any given $2 \times 2$ matrix is much faster then a larger
$n \times n$ matrix due to a computational shortcut. This looks like the following:

\begin{gather*}
  \text{let}\:A=
  \begin{bmatrix}
    a & b\\
    c & d\\
  \end{bmatrix}\\
  |A| = 
  \begin{vmatrix}
    a & b\\
    c & d\\
  \end{vmatrix}
  = ad - cb\\
  \text{Matrix is invertible, thus:}\\
  ad - cb \neq 0\\
  A^{-1} = \frac{1}{ad-cb}
  \begin{bmatrix}
    d & -b\\
    -c & a\\
  \end{bmatrix}\\
  A^{-1}A = \frac{1}{ad-cb}
  \begin{bmatrix}
    d & -b\\
    -c & a\\
  \end{bmatrix}
  \begin{bmatrix}
    a & b\\
    c & d\\
  \end{bmatrix}
  =
  \frac{1}{ad-cb}
  \begin{bmatrix}
    ad-bc & 0\\
    0 & ad-bc\\
  \end{bmatrix}
  =
  \begin{bmatrix}
    1 & 0\\
    0 & 1\\
  \end{bmatrix} = I_2 \qed
\end{gather*}


\subsection{Example algebraic manipultion of matrices}
let $A,B,C$ be invertible $n\times n$ matrices.\\
Does the $C^{-1}(A+x)B^{-1}=I_n$ have a solution?

\begin{gather*}
  CC^{-1}(A+x)B^{-1}B=CI_nB\\
  I_n(A+x)I_n=CB\\
  A+x = CB\\
  x = CB-A
\end{gather*}
Note that the order of multiplication is important since $AB \neq BA$.\\
\\
Compute the inverted matrix $A^{-1}$
\begin{gather*}
  A =
  \begin{bmatrix}
    1 & 2\\
    3 & 4\\
  \end{bmatrix}
  \qquad A^{-1} =
  \begin{bmatrix}
    a & b\\
    c & d\\
  \end{bmatrix}\\
  AA^{-1} =
  \begin{bmatrix}
    1 & 2\\
    3 & 4\\
  \end{bmatrix}
  \cdot
  \begin{bmatrix}
    a & b\\
    c& d\\
  \end{bmatrix}
  =
  \begin{bmatrix}
     1 & 0\\
     0 & 1\\
  \end{bmatrix}\\
  \text{To prevent having to do Gaussian elemination twice:}\\
  \begin{bmatrix}[cc|cc]
    1 & 2 & 1 & 0\\
    3 & 4 & 0 & 1\\
  \end{bmatrix}
  \sim
  \begin{bmatrix}[cc|cc]
    1 & 0 & -2 & 1\\
    0 & 1 & \frac{3}{2} & -\frac{1}{2}\\
  \end{bmatrix}
\end{gather*}

Note that this is of the form: $\begin{bmatrix}[c|c]A & I_n\\ \end{bmatrix} \sim \begin{bmatrix}[c|c]I_n & A^{^-1}\\ \end{bmatrix}$



\end{document}