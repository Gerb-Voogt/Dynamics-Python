\documentclass[11pt, a4paper]{article}

\usepackage{graphicx}
\usepackage[english]{babel}
\usepackage[utf8x]{inputenc}
\usepackage{amsmath}
\usepackage[a4paper,top=3cm,bottom=2cm,left=2cm,right=2cm,marginparwidth=1.75cm]{geometry}
\usepackage{amssymb}

\graphicspath{ {./images} }

\makeatletter
\renewcommand*\env@matrix[1][*\c@MaxMatrixCols c]{%
  \hskip -\arraycolsep
  \let\@ifnextchar\new@ifnextchar
  \array{#1}}
\makeatother

\begin{document}
\section{Lecture 1: Linear systems and echelon forms (11/02/2020)}
\subsection{Introduction}
Linear algebra is a branch of mathematics that deals with the analysis of linear systems
of equations. It has common applications in many branches of both physical sciences (Quantum mechanics, relativistic mechanics) as well
as computer science (graphics proccesing, machine learning) and economics.\\
\\
In general linear equations take the following form:
\begin{equation}
    a_1x_1 + a_2x_2 + \cdots + a_nx_n = b
\end{equation}
or in sigma notation:
\begin{equation}
    \sum_{i=1}^{n} a_ix_i = b
\end{equation}
Where $a_i$ are the coefficients and $x_i$ the variables.\\
\\
Any system of equations has either a single unique solution, no solution at all, 
or infinitly many solutions. Determening which of these is correct for any given system will be discussed later.

\subsection{Systems of equations and matrices}
Systems of equations can be very tedious to write out. It's also very difficult
to correctly manipulate the rows of a system of equations without making small mistakes.
Because of this the preffered method of writing systems of equations is in a $m\times n$ matrix,
such that the rows contain the equations and the columns the coefficients. A system of
equations and the augmented matrix associated with said system are represented below.
\\
\begin{align*}
    \begin{cases} 
        x_1 + 5x_2 + 3x_3 &= 1 \\
        2x_1 + x_2 + 15x_3 &= 8 \\
        -x_1 + x_2 + 3x_3 &= 1 \\
       \end{cases}\hspace{-1em}
    \qquad &\Leftrightarrow \quad
    \begin{bmatrix}[ccc|c]
       1 & 5 & 3 & 1\\
       2 & 1 & 15 & 8\\
       -1 & 1 & 3 & 1\\
    \end{bmatrix}      
\end{align*}
Any given system of equation can be manipulated with the 3 fundamental row operations.
These are as follows:
\begin{description}
    \item[First] Replacing a row with the sum of itself and another row
    \item[Second] Interchanging two rows
    \item[Third]  Multiplying a row by any non-zero constant
\end{description}
Application of these row operations to solve for variables is commonly referred to as
Gaussian elimination. The second and third rule are fairly self explanatory.
Rule 1 however is best demonstrated with an example.
\begin{align*}
    \begin{bmatrix}[cc|c]
        1 & 1 & 6\\
        2 & 3 & 8\\
    \end{bmatrix}\hspace{-1em}
    \qquad &\sim \quad
    \begin{bmatrix}[cc|c]
       1 & 1 & 6 \\
       0 & 1 & -4 \\
    \end{bmatrix}      
\end{align*}
Note that the first equation was multiplied by -2 and subsequently added to equation 2.
The second row now has the form of $x_2 = -4$. The answer for $x_1$ can now be determined via
backsubsitution. This gives the following solution: $(x_1, x_2) = (10, -4)$

\subsection{Consistency of systems}
A system of equations can either be consistent, meaning the system has atleast 1 solution, or 
inconsistent, meaning no solution for the system excists. A system can be found 
inconsistent if Gaussian elimination gives a row in the following form:
\begin{align*}
    \begin{bmatrix}[ccc|c]
        0 & 0 & 0 & c\\
    \end{bmatrix}
    \;, c \neq 0
\end{align*}
As common sense tells us this is impossible since all 0 coefficients can never produce
a result that is not equal to zero. The following form can also arise:
\begin{align*}
    \begin{bmatrix}[ccc|c]
        0 & 0 & 0 & 0\\
    \end{bmatrix}
\end{align*}
This system is not inconsistent since all 0 coefficients should always produce 0. This form arises from 
gaussian elemination when the system has a redundant equation, which is to say, an equation that does not
add any extra information. An example of this is two rows with the same equation written differently.\\
\\
Consistent systems are typically solved by row reduction and solving for all unknown variables if there is a unique
solution. When the system does not have a unique solution, but instead infinitly many the system is solved by expressing
all basic variables in terms of free variables. More on basic and free variables will be covered in the echelon form section.
\subsection{Echelon forms}
There are a couple conditions which need to be met for a matrix to be considered in echelon form. These are
as follows:
\begin{description}
    \item[First] All non-zero rows are above the rows of all zeros
    \item[Second] Each pivot is to the right of the previous pivot
    \item[Third] All entries below a pivot are zero.  
\end{description}
Applying these rules gives a matrix of the following form:
\begin{align*}
    \begin{bmatrix}[ccc|c]
        \blacksquare & * & * & *\\
        0 & \blacksquare & * & *\\
        0 & 0 & \blacksquare & *\\
    \end{bmatrix}
\end{align*}
$\blacksquare$ represent pivots and $*$ can be any random number.\\
\\
Echelon form matrices can also be further reduced to reduced echelon form. In this
case all the entries above a pivot are 0 and the pivots are all equal to 1. This gives
the following matrix:
\begin{align*}
    \begin{bmatrix}[ccc|c]
        1 & 0 & 0 & *\\
        0 & 1 & 0 & *\\
        0 & 0 & 1 & *\\
    \end{bmatrix}
\end{align*}
\\
When a column has no pivot in reduced echelon the variable is what is referred to as a free variable.
The system will have a solution for any random value of the free variables and thus, infinitly many solutions.
An example of this will be listed below:
\begin{align*}
    \begin{bmatrix}[ccccc|c]
        1 & 0 & 0 & 0 & 3 & 5\\
        0 & 1 & 0 & 0 & -4 & 1\\
        0 & 0 & 0 & 1 & 9 & 4\\
        0 & 0 & 0 & 0 & 0 & 0 \\
    \end{bmatrix} 
    \hspace{-1em}
    \qquad &\Rightarrow \quad
    \begin{cases} 
        x_1 &= 5 + 3x_5 \\
        x_2 &= 1 + 4x_5\\
        x_3 &= free\\
        x_4 &= 4 - 9x_5\\
        x_5 &= free\\
       \end{cases}     
\end{align*}

\end{document}
