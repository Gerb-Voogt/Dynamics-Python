\documentclass[11pt, a4paper]{article}

\usepackage{graphicx}
\usepackage[a4paper,top=3cm,bottom=2cm,left=2cm,right=2cm,marginparwidth=1.75cm]{geometry}
\usepackage[english]{babel}
\usepackage[utf8x]{inputenc}
\usepackage{subfig}
\usepackage{amsmath}
\usepackage{amssymb}
\usepackage{float}

\graphicspath{ {./images} }
\newcommand*{\qed}{\hfill\ensuremath{\quad\square}}%
\newcommand*{\rad}{\ensuremath{\,\text{rad}}}
\newcommand*{\R}{\ensuremath{\mathbb{R}}}

\makeatletter
\renewcommand*\env@matrix[1][*\c@MaxMatrixCols c]{%
  \hskip -\arraycolsep
  \let\@ifnextchar\new@ifnextchar
  \array{#1}}
\makeatother

\newtheorem{theorem}{Theorem}

%------------------------------------------------
%Templates for images and figures
% \begin{figure}[h]
%   \centering
%   \subfloat[caption 1]{{\includegraphics[width=30mm]{images/placeholder.png}}}%
%   \qquad
%   \subfloat[caption 2]{{\includegraphics[width=30mm]{images/placeholder.png}}}%
%   \caption{Description}
% \end{figure}

% \begin{figure}[h]
%   \centerline{\includegraphics[width=50mm]{images/placeholder.png}}
%   \caption{Description}
% \end{figure}
%-----------------------------------------------

\begin{document}

\setcounter{section}{13}
\section{Lecture 14: The least squares problems (27/03/2020)}
\subsection{Least square solution}
The least square solution is a method of solving inconsistent linear systems.
\begin{theorem}
  $A\vec{x}=\vec{b}$ is consistent iff $\vec{b}\in \text{Col}(A)$, since $\vec{b}$ can then be written as a linear combination of the columns of $A$.
\end{theorem}
Thus a system is inconsistent if $\vec{b} \notin \text{Col}(A)$. To solve an inconsistent linear system for the closest possible solution we solve for the vector $\hat{b}$ which is the projection of the vector $\vec{b}$ onto the column space of $A$\footnote{$\text{proj}_{\text{Col}(A)}(\vec{b})$}. The system then becomes:
\begin{equation}
  A\hat{x} = \hat{b}
\end{equation}
Note that the vector $\vec{x}$ is a different vector then the vector $\hat{x}$. $\vec{x}$ would be the actual solution to the system which we cannot find and $\hat{x}$ is the closest approximate solution to the system. Since the vector $\hat{x}$ is defined to be part of the column space of $A$ we know for a fact that this system is consistent.
\begin{figure}[h]
  \centerline{\includegraphics[width=70mm]{images/Projection.png}}
  \caption{Graphical representation of the projection of the vector $\vec{x}$ onto the column space of $A$ ($W = \text{Col}(A)$)}
\end{figure}
An interesting thing to note is the following theorem, which offers a different way of computing the least squares solution.
\begin{theorem}
  The following statements are equivelant:
  \begin{itemize}
    \item $\hat{x}$ is the LSQ solution to $A\vec{x} = \vec{b}$.
    \item $\hat{x}$ is the (actual) solution to the equation $A^TA\hat{x} = A^T\vec{b}$.
  \end{itemize}
\end{theorem}

\subsection{LSQ Error}
As mentioned before the solution to the system $A\hat{x}=\hat{b}$ is not an exact solution, but rather the closest possible solution. Not all approximate solutions are made equal. Some will have a large error while some have a smaller error. We can find the error by checking the distance from the vector $\vec{b}$ to the new projected vector $\hat{b}$. This distance represents how much the system had to be adjusted to make it consistent. When the distance is very large a big adjustment to the system had to be made to force consistency. If the distance is very small only a small adjustment had to be made to make the system consistent.
\begin{equation}
  \text{dist}(\vec{b},\hat{b}) = ||\vec{b}-\hat{b}|| = ||\vec{b} - A\hat{x}||
\end{equation}

\subsection{Example with linear regression}
\setcounter{equation}{0}
\begin{figure}[H]
  \centerline{\includegraphics[width=70mm]{images/Linear_regression.png}}
  \caption{The graph of the inconsistent linear system and the closest consisten solution}
\end{figure}
\begin{gather}
  \text{let:} A =
  \begin{bmatrix}
    1 & 1\\
    1 & 2\\
    1 & 3\\
  \end{bmatrix},\;
  \vec{x} = \begin{bmatrix} \beta_0 \\ \beta_1 \\ \end{bmatrix},\;
  \vec{b} = \begin{bmatrix} 2\\ 4\\ 0\\ \end{bmatrix}
\end{gather}
The column space of $A$ will then be:
\begin{gather}
  \text{Col}(A) = \Bigg\{ \begin{bmatrix} 1\\ 1\\ 1\\ \end{bmatrix}
                  \begin{bmatrix} 1\\ 2\\ 3\\ \end{bmatrix} \Bigg\}
\end{gather}
We now apply the Gram-Schmidt process to find an orthogonal basis for the column space of $A$:
\begin{gather}
  \vec{v}_1 = \vec{x}_1 = \begin{bmatrix} 1\\ 1\\ 1\\ \end{bmatrix}\\
  \vec{v}_2 = \vec{x}_2 - \frac{\langle \vec{x}_2 | \vec{v}_1 \rangle}{\langle \vec{v}_1 | \vec{v}_1 \rangle} = \begin{bmatrix} 1\\ 2\\ 3\\ \end{bmatrix} - \frac{6}{3} \begin{bmatrix} 1\\ 1\\ 1\\ \end{bmatrix} = \begin{bmatrix} -1\\ 0\\ 1\\ \end{bmatrix}
\end{gather}
The orthogonal basis for will then be $\{\vec{v}_1, \vec{v}_2 \}$ which is:
\begin{gather}
  \Bigg\{ 
    \begin{bmatrix} 1\\ 1\\ 1\\ \end{bmatrix},
    \begin{bmatrix} -1\\ 0\\ 1\\ \end{bmatrix} 
  \Bigg\}
\end{gather}
The projection of vector $b$ onto the column space of $A$ $\hat{b}$ will then be:
\begin{align}
  \text{proj}_{\text{Col}(A)}(\vec{b}) = \hat{b} &= \frac{\langle \vec{b} | \vec{v}_1 \rangle}{\langle \vec{v}_1 | \vec{v}_1 \rangle}\vec{v}_1 + \frac{\langle \vec{b} | \vec{v}_2 \rangle}{\langle \vec{v}_2 | \vec{v}_2 \rangle}\vec{v}_2 \notag\\
  \hat{b} &= \frac{6}{3}\begin{bmatrix} 1\\ 1\\ 1\\ \end{bmatrix} + \frac{-2}{2}\begin{bmatrix} -1\\ 0\\ 1\\ \end{bmatrix} \notag\\ 
  \hat{b} &= \begin{bmatrix} 3\\ 2\\ 1\\ \end{bmatrix}
\end{align}
The LSQ solution to the problem will then become:
\begin{gather}
  A\hat{x} = \hat{b} \Rightarrow
  \begin{bmatrix}[cc|c]
    1 & 1 & 3\\
    1 & 2 & 2\\
    1 & 3 & 1\\
  \end{bmatrix}
  \sim
  \begin{bmatrix}[cc|c]
    1 & 1 & 3\\
    0 & 1 & -1\\
    0 & 2 & -2\\
  \end{bmatrix}
  \sim
  \begin{bmatrix}[cc|c]
    1 & 0 & 4\\
    0 & 1 & -1\\
    0 & 0 & 0\\
  \end{bmatrix}\\
  \text{thus: } \hat{\beta}_0 = 4, \; \hat{\beta}_1 = -1 \Rightarrow 
  \hat{x} = \begin{bmatrix} \hat{\beta_0}\\ \hat{\beta_1}\\ \end{bmatrix} = \begin{bmatrix} 4\\ -1\\ \end{bmatrix}
\end{gather}
Where equation (8) is the closest consistent solution for the linear system.
\end{document}